\documentclass[11pt]{article}

\usepackage{calc}
\usepackage{pdfpages}

\usepackage[T1]{fontenc}
%\usepackage[sc]{mathpazo}

\usepackage{libertine}
\usepackage[override]{cmtt}

\usepackage[style=ieee,block=ragged,doi=false,url=false,sorting=ydnt]{biblatex}
\addbibresource{Zhang_0022:Fan.bib}

% Layout: Puts the section titles on left side of page
\reversemarginpar

\usepackage[paper=letterpaper,
%includefoot, % Uncomment to put page number above margin
marginparwidth=1.1in,     % Length of section titles
marginparsep=.05in,       % Space between titles and text
margin=1in,               % 1 inch margins
includemp]{geometry}

%% More layout: Get rid of indenting throughout entire document
\setlength{\parindent}{0in}

%% This gives us fun enumeration environments. compactitem will be nice.
\usepackage{paralist}
\usepackage{verbatim}

%% Reference the last page in the page number
%
% NOTE: comment the +LP line and uncomment the -LP line to have page
%       numbers without the ``of ##'' last page reference)
%
% NOTE: uncomment the \pagestyle{empty} line to get rid of all page
%       numbers (make sure includefoot is commented out above)
%
\usepackage{fancyhdr,lastpage}
\pagestyle{fancy}
%\pagestyle{empty}      % Uncomment this to get rid of page numbers
\fancyhf{}\renewcommand{\headrulewidth}{0pt}
\fancyfootoffset{\marginparsep+\marginparwidth}
\newlength{\footpageshift}
\setlength{\footpageshift}
{0.5\textwidth+0.5\marginparsep+0.5\marginparwidth-2in}
\lfoot{\hspace{\footpageshift}%
\parbox{4in}{\, \hfill %
\arabic{page} of \protect\pageref*{LastPage} % +LP
%                    \arabic{page}                               % -LP
\hfill \,}}

% Finally, give us PDF bookmarks
\usepackage{color,hyperref}
\definecolor{darkblue}{rgb}{0.0,0.0,0.3}
\hypersetup{colorlinks,breaklinks,
linkcolor=darkblue,urlcolor=darkblue,
anchorcolor=darkblue,citecolor=darkblue}



\newcommand{\ignore}[1]{}
%%%%%%%%%%%%%%%%%%%%%%%% End Document Setup %%%%%%%%%%%%%%%%%%%%%%%%%%%%


%%%%%%%%%%%%%%%%%%%%%%%%%%% Helper Commands %%%%%%%%%%%%%%%%%%%%%%%%%%%%

% The title (name) with a horizontal rule under it
% (optional argument typesets an object right-justified across from name
%  as well)
%
% Usage: \makeheading{name}
%        OR
%        \makeheading[right_object]{name}
%
% Place at top of document. It should be the first thing.
% If ``right_object'' is provided in the square-braced optional
% argument, it will be right justified on the same line as ``name'' at
% the top of the CV. For example:
%
%       \makeheading[\emph{Curriculum vitae}]{Your Name}
%
% will put an emphasized ``Curriculum vitae'' at the top of the document
% as a title. Likewise, a picture could be included:
%
%   \makeheading[\includegraphics[height=1.5in]{my_picutre}]{Your Name}
%
% the picture will be flush right across from the name.
\newcommand{\makeheading}[2][]%
{\hspace*{-\marginparsep minus \marginparwidth}%
\begin{minipage}[t]{\textwidth+\marginparwidth+\marginparsep}%
  {\huge {#2} \hfill #1}\\[-0.5\baselineskip]%
\rule{\columnwidth}{1pt}%
\end{minipage}}

% The section headings
%
% Usage: \Section{section name}
%
% Follow this section IMMEDIATELY with the first line of the section
% text. Do not put whitespace in between. That is, do this:
%
%       \Section{My Information}
%       Here is my information.
%
% and NOT this:
%
%       \Section{My Information}
%
%       Here is my information.
%
% Otherwise the top of the section header will not line up with the top
% of the section. Of course, using a single comment character (%) on
% empty lines allows for the function of the first example with the
% readability of the second example.
\newcommand{\Section}[2]%
{\pagebreak[3]\vspace{.8\baselineskip}%
\phantomsection\addcontentsline{toc}{section}{#1}%
\hspace{0in}%
\marginpar{
%\raggedright \bfseries{\scshape{#1}}}#2}
\raggedright \scshape{#1}}#2}

% subsection with/without date
\newcommand{\Subsection}[2]{\textbf{#1}\hfill{#2}}
\newcommand{\SubsectionOne}[1]{\Subsection{#1}{}}

% An itemize-style list with lots of space between items
\newenvironment{outerlist}[1][\enskip\textbullet]%
{\begin{itemize}[#1]}{\end{itemize}%
\vspace{-.6\baselineskip}}

% An environment IDENTICAL to outerlist that has better pre-list spacing
% when used as the first thing in a \Section
\newenvironment{lonelist}[1][\enskip\textbullet]%
{\vspace{-\baselineskip}\begin{list}{#1}{%
\setlength{\partopsep}{0pt}%
\setlength{\topsep}{0pt}}}
{\end{list}\vspace{-.6\baselineskip}}

% An itemize-style list with little space between items
\newenvironment{innerlist}[1][\enskip\textbullet]%
{\begin{compactitem}[#1]}{\end{compactitem}}

% An environment IDENTICAL to innerlist that has better pre-list spacing
% when used as the first thing in a \Section
\newenvironment{loneinnerlist}[1][\enskip\textbullet]%
{\vspace{-\baselineskip}\begin{compactitem}[#1]}
{\end{compactitem}\vspace{-.6\baselineskip}}

% To add some paragraph space between lines.
% This also tells LaTeX to preferably break a page on one of these gaps
% if there is a needed pagebreak nearby.
\newcommand{\blankline}{\quad\pagebreak[3]}
\newcommand{\halfblankline}{\quad\vspace{-0.5\baselineskip}\pagebreak[3]}

% For \email{ADDRESS}, links ADDRESS to the url mailto:ADDRESS
\providecommand*\email[1]{\href{mailto:#1}{\tt #1}}


%%%%%%%%%%%%%%%%%%%%%%%% End Helper Commands %%%%%%%%%%%%%%%%%%%%%%%%%%%

%%%%%%%%%%%%%%%%%%%%%%%%% Begin CV Document %%%%%%%%%%%%%%%%%%%%%%%%%%%%

\begin{document}
\makeheading{Fan Zhang}

\newlength{\rcollength}\setlength{\rcollength}{1.65in}%

\Section{Contact Information}
\begin{tabular}[t]{@{}p{\textwidth-\rcollength}p{\rcollength}}
2 West Loop Road & \href{http://fanzhang.me}{\tt http://fanzhang.me}\\
New York, NY 10044  & \email{fanz@cs.cornell.edu}
\end{tabular}

\Section{Education}
\Subsection{Ph.D. Candidate in Computer Science}{Aug, 2014--present}
\begin{innerlist}
\item[] Advisor: Prof. Ari Juels
\item[] Dept.\ of Computer Science \\ Cornell University
\end{innerlist}

\smallskip
\Subsection{B.S. in Electronic Engineering}{Aug, 2010 -- Jul, 2014}
\begin{innerlist}
\item[] Tsinghua University, Beijing, China
\end{innerlist}

\Section{Research Areas}
Applied Cryptography, Trusted Hardware, Blockchain

\Section{Industry Adoption}
My research has led to direct industry adoption.
Town Crier~\cite{DBLP:conf/ccs/ZhangCCJS16} was licensed from Cornell by \href{https://chain.link}{Chainlink} and Ekiden~\cite{DBLP:conf/eurosp/ChengZKHHJJ0S19} is used in \href{https://oasislabs.com}{Oasis Labs}' products.


\Section{Honors/Awards}
\vspace{-5mm}
\begin{innerlist}
\item {IBM PhD Fellowship Award}\hfill {2018-2020}
\item {Academic Excellence Scholarship, Tsinghua University, China}\hfill{2013}
\item {National Scholarship, the Ministry of Education of China}\hfill{2012}
\item {Freshman Scholarship, Tsinghua University, China}\hfill{2010}
\end{innerlist}


\Section{Selected Publications}
\vspace{-6mm}

\nocite{*}
\printbibliography[heading=none]


\Section{Professional Activity}
\vspace{-5mm}
\begin{innerlist}
\item {\bf Program Committee}: BITCOIN'18, collocated with Financial Crypto 2018.
\item {\bf Reviewer}: ACM Computing Surveys (2018), Nature Sustainability (2018)
\item {\bf Subreviewer}: USENIX Security (2016), TCC (2019)
\end{innerlist}


\Section{Software Artifacts}
My research yields practical systems and production-ready software artifacts.
Here is a selected list of them and please see my Github page for more.

\begin{innerlist}
\item
Town Crier: an Authenticated Data Feed For Smart Contracts \\
\url{https://town-crier.org}

\item
CHURP: Dynamic-Committee Proactive Secret Sharing \\
\url{https://churp.io}

\item
mbedtls-SGX: a SGX-friendly TLS stack (ported from mbedtls)\\
\url{https://github.com/bl4ck5un/mbedtls-SGX}
\end{innerlist}

\Section{Working Experience}
\Subsection{Researcher}{May, 2017 -- Aug, 2017} \\
Oasis Labs \hfill Berkeley, CA

\Subsection{Researcher} {Jul, 2017 -- Aug, 2017} \\
SPR (Security \& Privacy Research), Intel Labs \hfill Hillsboro, OR

\Subsection{System developer intern} {Jun, 2013 -- May, 2014} \\
Intel Opensource Technology Center (01.org) \hfill Beijing, China

\Section{Teaching Experience}
\vspace{-5mm}
\begin{innerlist}
\item TA for CS5435: Security and Privacy in the Wild \hfill 2015, Fall
\item TA for CS5300: the Architecture of Large-scale Information Systems \hfill 2015, Spring
\item TA for CS4410: Operating Systems \hfill 2014 Fall
\end{innerlist}

\Section{Invited Talks}
\SubsectionOne{On Trusted Hardware and Blockchain Hybridization}
\begin{innerlist}
\item Northeastern University, Cybersecurity Speaker Series.\hfill Jan, 2019
\item MIT, CSAIL.\hfill Nov, 2018
\item New York University, CS Colloquium.\hfill Oct, 2018
\end{innerlist}
\SubsectionOne{Paralysis Proof}
\begin{innerlist}
\item IC3 Retreat, New York City. \hfill May, 2018
\item 5th Bitcoin Workshop, Financial Crypto'18, Curacao. \hfill{Mar, 2018}
\end{innerlist}

\SubsectionOne{REM}
\begin{innerlist}
\item
USENIX Security'17, Vancouver BC, Canada. \hfill{Aug, 2017}
\end{innerlist}

\SubsectionOne{Town Crier}
\begin{innerlist}
\item
Silicon Valley Ethereum Meetup, Santa Clara, CA. \hfill{Aug, 2017}
\item
IC3 Retreat, San Francisco, CA. \hfill{Mar, 2017}
\item
CCS'16, Vienna, Austria. \hfill{Oct, 2016}
\item
IC3 Retreat, New York City. \hfill{May, 2016}
\end{innerlist}

\end{document}

%----------------------------------------------------------------------%
% The following is copyright and licensing information for
% redistribution of this LaTeX source code; it also includes a liability
% statement. If this source code is not being redistributed to others,
% it may be omitted. It has no effect on the function of the above code.
%----------------------------------------------------------------------%
% Copyright (c) 2007, 2008, 2009, 2010, 2011 by Theodore P. Pavlic
%
% Unless otherwise expressly stated, this work is licensed under the
% Creative Commons Attribution-Noncommercial 3.0 United States License. To
% view a copy of this license, visit
% http://creativecommons.org/licenses/by-nc/3.0/us/ or send a letter to
% Creative Commons, 171 Second Street, Suite 300, San Francisco,
% California, 94105, USA.
%
% THE SOFTWARE IS PROVIDED "AS IS", WITHOUT WARRANTY OF ANY KIND, EXPRESS
% OR IMPLIED, INCLUDING BUT NOT LIMITED TO THE WARRANTIES OF
% MERCHANTABILITY, FITNESS FOR A PARTICULAR PURPOSE AND NONINFRINGEMENT.
% IN NO EVENT SHALL THE AUTHORS OR COPYRIGHT HOLDERS BE LIABLE FOR ANY
% CLAIM, DAMAGES OR OTHER LIABILITY, WHETHER IN AN ACTION OF CONTRACT,
% TORT OR OTHERWISE, ARISING FROM, OUT OF OR IN CONNECTION WITH THE
% SOFTWARE OR THE USE OR OTHER DEALINGS IN THE SOFTWARE.
%----------------------------------------------------------------------%
