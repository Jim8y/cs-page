\usepackage{calc}
\usepackage{pdfpages}

\usepackage[T1]{fontenc}
%\usepackage[tt=false]{libertine}

\usepackage[style=ieee,
doi=false,
url=false,
giveninits=false,
sorting=ydndt]{biblatex}


% special sort style
\DeclareSortingTemplate{ydndt}{
  \sort[direction=descending]{\field{year}}
  \sort[direction=descending]{\field{name}}
  \sort{\field{title}}
}

\usepackage{xpatch}
\addbibresource{Zhang_0022:Fan.bib}

% to make name bold
% taken from https://tex.stackexchange.com/questions/274436/highlight-an-author-in-bibliography-using-biblatex-allowing-bibliography-style-t/274571#274571

\newcommand*{\doboldhashes}[1]{%
  \iffieldequalstr{hash}{#1}
    {\bfseries\listbreak}
    {}}%

\newbibmacro*{name:bold}{%
  \forlistloop{\doboldhashes}{\boldnames}%
}

\newcommand*{\boldnames}{}

\xpretobibmacro{name:family}{\begingroup\usebibmacro{name:bold}}{}{}
\xpretobibmacro{name:given-family}{\begingroup\usebibmacro{name:bold}}{}{}
\xpretobibmacro{name:family-given}{\begingroup\usebibmacro{name:bold}}{}{}
\xpretobibmacro{name:delim}{\begingroup\normalfont}{}{}

\xapptobibmacro{name:family}{\endgroup}{}{}
\xapptobibmacro{name:given-family}{\endgroup}{}{}
\xapptobibmacro{name:family-given}{\endgroup}{}{}
\xapptobibmacro{name:delim}{\endgroup}{}{}

\renewcommand*{\boldnames}{}
\forcsvlist{\listadd\boldnames}
  {ae021641a8f85fd2607152ab99ed115b,a66874965c0ba479de19bb40f33388d1}
  
% end of macros for making names bold

% Layout: Puts the section titles on left side of page
\reversemarginpar

\usepackage[paper=letterpaper,
%includefoot, % Uncomment to put page number above margin
marginparwidth=.9in,     % Length of section titles
marginparsep=.05in,       % Space between titles and text
%margin=.8in,               % 1 inch margins
top=2cm,
left=2.5cm,
right=2.5cm,
includemp]{geometry}

%% More layout: Get rid of indenting throughout entire document
\setlength{\parindent}{0in}

%% This gives us fun enumeration environments. compactitem will be nice.
\usepackage{paralist}
\usepackage{verbatim}

%% Reference the last page in the page number
%
% NOTE: comment the +LP line and uncomment the -LP line to have page
%       numbers without the ``of ##'' last page reference)
%
% NOTE: uncomment the \pagestyle{empty} line to get rid of all page
%       numbers (make sure includefoot is commented out above)
%
\usepackage{fancyhdr,lastpage}
\pagestyle{fancy}
%\pagestyle{empty}      % Uncomment this to get rid of page numbers
\fancyhf{}\renewcommand{\headrulewidth}{0pt}
\fancyfootoffset{\marginparsep+\marginparwidth}
\newlength{\footpageshift}
\setlength{\footpageshift}
{0.5\textwidth+0.5\marginparsep+0.5\marginparwidth-2in}
\lfoot{\hspace{\footpageshift}%
\parbox{4in}{\, \hfill %
\arabic{page} of \protect\pageref*{LastPage} % +LP
%                    \arabic{page}                               % -LP
\hfill \,}}

% Finally, give us PDF bookmarks
\usepackage{color,hyperref}
\hypersetup{colorlinks,breaklinks,allcolors=blue}


\newcommand{\ignore}[1]{}
%%%%%%%%%%%%%%%%%%%%%%%% End Document Setup %%%%%%%%%%%%%%%%%%%%%%%%%%%%


%%%%%%%%%%%%%%%%%%%%%%%%%%% Helper Commands %%%%%%%%%%%%%%%%%%%%%%%%%%%%

% The title (name) with a horizontal rule under it
% (optional argument typesets an object right-justified across from name
%  as well)
%
% Usage: \makeheading{name}
%        OR
%        \makeheading[right_object]{name}
%
% Place at top of document. It should be the first thing.
% If ``right_object'' is provided in the square-braced optional
% argument, it will be right justified on the same line as ``name'' at
% the top of the CV. For example:
%
%       \makeheading[\emph{Curriculum vitae}]{Your Name}
%
% will put an emphasized ``Curriculum vitae'' at the top of the document
% as a title. Likewise, a picture could be included:
%
%   \makeheading[\includegraphics[height=1.5in]{my_picutre}]{Your Name}
%
% the picture will be flush right across from the name.
\newcommand{\makeheading}[2][]%
{\hspace*{-\marginparsep minus \marginparwidth}%
\begin{minipage}[t]{\textwidth+\marginparwidth+\marginparsep}%
  {\Huge \bf {\fontfamily{ppl}\selectfont #2} \hfill #1}\\[-0.3\baselineskip]%
\rule{\columnwidth}{1pt}%
\end{minipage}}

% The section headings
%
% Usage: \Section{section name}
%
% Follow this section IMMEDIATELY with the first line of the section
% text. Do not put whitespace in between. That is, do this:
%
%       \Section{My Information}
%       Here is my information.
%
% and NOT this:
%
%       \Section{My Information}
%
%       Here is my information.
%
% Otherwise the top of the section header will not line up with the top
% of the section. Of course, using a single comment character (%) on
% empty lines allows for the function of the first example with the
% readability of the second example.
\newcommand{\Section}[2]%
{\pagebreak[3]\vspace{.8\baselineskip}%
\phantomsection\addcontentsline{toc}{section}{#1}%
\hspace{0in}%
\marginpar{
%\raggedright \bfseries{\scshape{#1}}}#2}
\raggedright \scshape{#1}}#2}

% subsection with/without date
\newcommand{\Subsection}[2]{\textbf{#1}\hfill{#2}}
\newcommand{\SubsectionOne}[1]{\Subsection{#1}{}}

% An itemize-style list with lots of space between items
\newenvironment{outerlist}[1][\enskip\textbullet]%
{\begin{itemize}[#1]}{\end{itemize}%
\vspace{-.6\baselineskip}}

% An environment IDENTICAL to outerlist that has better pre-list spacing
% when used as the first thing in a \Section
\newenvironment{lonelist}[1][\enskip\textbullet]%
{\vspace{-\baselineskip}\begin{list}{#1}{%
\setlength{\partopsep}{0pt}%
\setlength{\topsep}{0pt}}}
{\end{list}\vspace{-.6\baselineskip}}

% An itemize-style list with little space between items
\newenvironment{innerlist}[1][\enskip\textbullet]%
{\begin{compactitem}[#1]}{\end{compactitem}}

% An environment IDENTICAL to innerlist that has better pre-list spacing
% when used as the first thing in a \Section
\newenvironment{loneinnerlist}[1][\enskip\textbullet]%
{\vspace{-\baselineskip}\begin{compactitem}[#1]}
{\end{compactitem}\vspace{-.1\baselineskip}}

% To add some paragraph space between lines.
% This also tells LaTeX to preferably break a page on one of these gaps
% if there is a needed pagebreak nearby.
\newcommand{\blankline}{\quad\pagebreak[3]}
\newcommand{\halfblankline}{\quad\vspace{-0.5\baselineskip}\pagebreak[3]}

% For \email{ADDRESS}, links ADDRESS to the url mailto:ADDRESS
\providecommand*\email[1]{\href{mailto:#1}{\tt #1}}


%%%%%%%%%%%%%%%%%%%%%%%% End Helper Commands %%%%%%%%%%%%%%%%%%%%%%%%%%%
